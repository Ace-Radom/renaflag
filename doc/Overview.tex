\documentclass[12pt , a4paper , oneside]{ctexart}
\pagenumbering{roman}

\usepackage{geometry}
\geometry{
    left=2.28cm, 
    right=2.28cm, 
    top=4cm, 
    bottom=4cm
}

\usepackage{listings}

\usepackage{color}
\usepackage{xcolor}
\usepackage{framed}
\definecolor{file}{rgb}{0.67,0.31,0.32}
\definecolor{program}{RGB}{196,132,192}

\lstset{
    columns=fixed,       
    numbers=left,                                        % 在左侧显示行号
    frame=none, %tb,                                     % 不显示背景边框
    backgroundcolor=\color[RGB]{244,244,244},            % 设定背景颜色:浅晦涩,近透明
    keywordstyle=\color[RGB]{40,40,255},                 % 设定关键字颜色
    numberstyle=\tiny\color{darkgray},                   % 设定行号格式
    commentstyle=\it\color[RGB]{0,96,96},                % 设置代码注释的格式
    stringstyle=\rmfamily\slshape\color[RGB]{128,0,0},   % 设置字符串格式
    showstringspaces=false,                              % 不显示字符串中的空格
    language=C++,                                        % 设置语言C, C++, bash, Fortran, Python, TeX, make
    basicstyle=\small, %\scriptsize%\tiny%\footnotesize  % basic fontsize in it
    breaklines=true
}

\title{renaflag库总览}
\author{Ace\_Radom}
\date{2022.12}

\begin{document}

    \maketitle

    \newpage

    \section{概述}

        renaflag是一个C++第三方轻量级命令行解析库
        
        目前在Windows内使用MinGW完成开发, 基于C++11标准

        在日后renaflag可能会被移植兼容linux或同平台的VC编译器

    \newpage

    \section{安装}

        renaflag仅包含一个头文件\textcolor{file}{renaflag.h}和一个源程序\textcolor{file}{renaflag.cpp}

        普通使用可以直接拷贝源码, 放置于需要使用该库的程序的同文件夹内并连接

        同时renaflag也提供CMakeLists, 配合C++环境编译为动态或静态库

        比如在MinGW环境下, 只需在renaflag源码根目录下执行:

        \textit{\ \ \ \ \ \ cmake . -G "MinGW Makefiles"}

        \textit{\ \ \ \ \ \ mingw32-make}
        
        \textit{\ \ \ \ \ \ mingw32-make install}

        如此就能安装renaflag的动态和静态库, 随后只需在项目的CMakeLists中连接即可
    
    \newpage

    \section{基本使用}

        \small
        
        注:若没有特殊说明 本文中提到的“标准拉丁字母”皆代表自a至z的26个英文字母
        
        \dotfill
        \normalsize
        
        添加renaflag库的头文件只需在项目中主程序/头文件头部加上

        \begin{lstlisting}[numbers=none]
        #include"renaflag.h"
        \end{lstlisting}

        在头文件\textcolor{file}{renaflag.h}中定义了\textcolor{program}{renaflag}类, 封装于命名空间\textcolor{program}{rena}下

        这是renaflag实现命令行解析的基类

        \subsection{初始化}

            在\textcolor{program}{renaflag}类内声明了未被实现的成员函数\textcolor{program}{preset}

            所有的命令行标识符声明, 帮助文档声明等预设选项都应该被包含在使用renaflag的源程序内对\textcolor{program}{preset}成员函数的实现代码内

            因此, 初始化renaflag的完整过程为:

            \begin{lstlisting}[numbers=none]
        #include"renaflag.h"

        void rena::renaflag::preset(){
            ... // 声明标识符

            ... // 声明帮助文档 (可选)
        }

        rena::renaflag rf;
            \end{lstlisting}

            类\textcolor{program}{renaflag}的构析函数会自动调用\textcolor{program}{preset}成员函数, 无需在程序内再次初始化



\end{document}